%%%%%%%%%%%%%%%%%%%%%%%%%%%%%%%%%%%%%%%%%
% Classicthesis-Styled CV
% LaTeX Template
% Version 1.0 (22/2/13)
%
% This template has been downloaded from:
% http://www.LaTeXTemplates.com
%
% Original author:
% Alessandro Plasmati
%
% License:
% CC BY-NC-SA 3.0 (http://creativecommons.org/licenses/by-nc-sa/3.0/)
%
%%%%%%%%%%%%%%%%%%%%%%%%%%%%%%%%%%%%%%%%%

%-------------------------------------------------------------------------------
%	PACKAGES AND OTHER DOCUMENT CONFIGURATIONS
%-------------------------------------------------------------------------------

\documentclass{scrartcl}

\reversemarginpar % Move the margin to the left of the page

\newcommand{\MarginText}[1]{\marginpar{\raggedleft\itshape\small#1}}
  % New command defining the margin text style

\usepackage[nochapters]{classicthesis}
  % Use the classicthesis style for the style of the document
\usepackage[LabelsAligned]{currvita}
  % Use the currvita style for the layout of the document

\renewcommand{\cvheadingfont}{\LARGE\color{Maroon}}
  % Font color of your name at the top

\usepackage{hyperref} % Required for adding links and customizing them
\hypersetup{colorlinks, breaklinks, urlcolor=Maroon, linkcolor=Maroon}
  % Set link colors

\newlength{\datebox}\settowidth{\datebox}{Spring 2011}
  % Set the width of the date box in each block

\newcommand{\NewEntry}[3]{\noindent\hangindent=2em\hangafter=0
  \parbox{\datebox}{\small \textit{#1}}\hspace{1.5em} #2 #3
  % Define a command for each new block - change spacing and font sizes here:
  % #1 is the left margin, #2 is the italic date field and #3 is the
  % position/employer/location field

  \vspace{0.5em} % Add some white space after each new entry
}

\newcommand{\Description}[1]{\hangindent=2em\hangafter=0\noindent\raggedright
  \footnotesize{#1}\par\normalsize\vspace{1em}}
  % Define a command for descriptions of each entry - change spacing and font
  % sizes here

\date{}

%-------------------------------------------------------------------------------

\begin{document}

\thispagestyle{empty} % Stop the page count at the bottom of the first page

%-------------------------------------------------------------------------------
%	NAME AND CONTACT INFORMATION SECTION
%-------------------------------------------------------------------------------

\begin{cv}{\spacedallcaps{Cristian Adri\'an Cardellino}}\vspace{1.5em} % Your name

\noindent\spacedlowsmallcaps{\Large{Datos Personales}}\vspace{0.5em}

\NewEntry{DNI}{34335379} % DNI

\NewEntry{Nacimiento}{29 de Abril de 1989}

\NewEntry{T\'itulo}{Dr.\ en Cs.\ de la Computaci\'on}

% \NewEntry{Ingreso}{2013}

% \NewEntry{Egreso}{2018}

% \NewEntry{Promedio}{8.74}

\vspace{2em}

% \noindent\spacedlowsmallcaps{Goal}\vspace{1em} % Goal heading, could be used
% for a quotation or short profile instead

% \Description{Gain fundamental experience in my area of interest and
% expertise.}\vspace{2em} % Goal text

%-------------------------------------------------------------------------------
%	ANTECEDENTES DOCENTES
%-------------------------------------------------------------------------------

\noindent\spacedlowsmallcaps{\Large{Docencia}}\vspace{1em}

\NewEntry{4 a\~nos}{\textsc{Profesor Ayudante A} --- FAMAFyC. UNC.}

\Description{\MarginText{2017,2019}{Bases de Datos -- 2 Semestres.
C\'ordoba, Argentina.}}

\Description{\MarginText{2019}{Miner\'ia de Datos para Texto -- 1 Semestre.
C\'ordoba, Argentina.}}

\Description{\MarginText{2015-2019}{Paradigmas de Programaci\'on -- 5
Semestres. C\'ordoba, Argentina.}}

\Description{\MarginText{2018}{Sistemas Operativos -- 1
Semestre. C\'ordoba, Argentina.}}

\Description{\MarginText{2015-2016}{Ingenier\'ia del Software I -- 2 Semestres.
C\'ordoba, Argentina.}}

%------------------------------------------------

\NewEntry{2.5 a\~nos}{\textsc{Ayudante de Alumno}  --- FAMAFyC. UNC.}

\Description{\MarginText{2011, 2013}{Paradigmas de Programaci\'on -- 2
Semestres. C\'ordoba, Argentina.}}

\Description{\MarginText{2012}{Redes y Sistemas Distribu\'idos -- 1 Semestre.
C\'ordoba, Argentina.}}

\Description{\MarginText{2011-2012}{Ingenier\'ia del Software I -- 2 Semestres.
C\'ordoba, Argentina.}}

%------------------------------------------------

\NewEntry{}{\textsc{Suplencias}}

\Description{\MarginText{2017}{Miner\'ia de datos para texto -- 2 Semanas.
FAMAFyC.  UNC. C\'ordoba, Argentina.}}

%------------------------------------------------

\NewEntry{}{\textsc{Cursos y Tutoriales}}

\Description{\MarginText{2018-2019}{Aprendizaje Profundo -- Curso Te\'orico-Pr\'actico. 
Diplomatura en ciencias de datos, aprendizaje autom\'atico y sus aplicaciones. FAMAFyC. UNC. C\'ordoba, Argentina.}}

\Description{\MarginText{2019}{Aprendizaje Autom\'atico Supervisado -- Curso Te\'orico-Pr\'actico. 
Diplomatura en ciencias de datos, aprendizaje autom\'atico y sus aplicaciones. FAMAFyC. UNC. C\'ordoba, Argentina.}}

\Description{\MarginText{2018}{Introducci\'on al Aprendizaje Autom\'atico -- Curso Te\'orico-Pr\'actico. 
Diplomatura en ciencias de datos, aprendizaje autom\'atico y sus aplicaciones. FAMAFyC. UNC. C\'ordoba, Argentina.}}

\Description{\MarginText{2018}{An\'alisis y visualizaci\'on de datos -- Tutor\'ias. Diplomatura en
ciencias de datos, aprendizaje autom\'atico y sus aplicaciones. FAMAFyC. UNC. C\'ordoba, Argentina.}}

\Description{\MarginText{2017}{Train and visualize a model in Tensorflow
-- Tutorial. XXV Escuela de Verano de Ciencias Inform\'aticas-RIO 2018. UNRC. R\'io Cuarto, Argentina.}}

\Description{\MarginText{2017}{Express Deep Learning with Python --
Tutorial. $1^{\circ}$ Escuela Argentina de Inteligencia Artificial. 
$46^{\circ}$ JAIIO. UTN. C\'ordoba. Argentina}}

%------------------------------------------------
\newpage

\NewEntry{}{\textsc{Formaci\'on de Recursos Humanos}}

\noindent\spacedlowsmallcaps{Finalizados}

\Description{\MarginText{2019}{Direcci\'on de Tesis de Grado de Santiago Marro: 
``Estudio de redes neuronales en escalera como m\'etodo semi-supervisado para 
reconocimiento de entidades nombradas en textos legales''.}}

\noindent\spacedlowsmallcaps{En Proceso}

\Description{\MarginText{2019}{Direcci\'on de Tesis de Grado de Emiliano Kokic: 
``Reconocimiento semi-supervisado de entidades nombradas mediante redes convolucionales 
en escalera''.}}

\Description{\MarginText{2019}{Direcci\'on de Tesis de Grado de Karen Haag: 
``Reconocimiento de entidades nombradas en texto de dominio legal''.}}

%------------------------------------------------

\NewEntry{}{\textsc{Tribunales de tesis}}

\Description{\MarginText{2019}{Tribunal de tesina de grado de David González: 
``Miner\'ia de Argumentos con Aprendizaje Profundo y Atenci\'on''.}}

\Description{\MarginText{2019}{Tribunal de tesina de grado de Lucas Garay: 
``Procesamiento de Im\'agenes M\'edicas para Generaci\'on Autom\'atica de Reportes''.}}

\Description{\MarginText{2019}{Tribunal de tesina de grado de Mauricio Mazuecos: 
``Estudio de Simplificaci\'on de Oraciones con Modelos Actor-Critic''.}}

\Description{\MarginText{2018}{Tribunal de tesina de grado de Juan Scavuzzo: 
``Estratificaci\'on temporal de Aedes Aegypti basada en herramientas geoespaciales y aprendizaje
autom\'atico''.}}

\Description{\MarginText{2018}{Tribunal de tesina de grado de Agust\'in Capello: 
``Recomendaci\'on de textos legales''.}}

\Description{\MarginText{2018}{Tribunal de tesina de grado de Agust\'in M\'arquez Braconi: 
``Framework para aprendizaje activo''.}}

\Description{\MarginText{2018}{Tribunal de tesina de grado de Jonathan David Mutal: 
``Evaluaci\'on comparativa de modelos de traducci\'on estad\'istica y neuronal''.}}


%------------------------------------------------

\NewEntry{}{\textsc{Dise\~no de Material Te\'orico-Pr\'actico}}

\Description{\MarginText{2017,2019}{Material te\'orico-pr\'actico clases de
miner\'ia de datos para texto.}}

\Description{\MarginText{2018-2019}{Material te\'orico-pr\'actico para clases de
aprendizaje profundo.}}

\Description{\MarginText{2019}{Material te\'orico-pr\'actico para clases de
aprendizaje autom\'atico supervisado.}}

\Description{\MarginText{2016-2019}{Laboratorios Paradigmas de Programaci\'on:
Orientaci\'on a objectos y frameworks.}}

\Description{\MarginText{2018}{Material te\'orico-pr\'actico para clases de
introducci\'on al aprendizaje autom\'atico.}}

\Description{\MarginText{2017}{Material te\'orico-pr\'actico tutoriales EAIA y
PyData.}}

\Description{\MarginText{2016}{Dise\~no de proyecto Ingenier\'ia del Software
I.}}

% \vspace{2em} % Extra space between major sections
\newpage

%-------------------------------------------------------------------------------
%	INVESTIGACIÓN
%-------------------------------------------------------------------------------

\noindent\spacedlowsmallcaps{\Large{Investigaci\'on}}\vspace{1em}

\noindent\spacedlowsmallcaps{Trabajos publicados en revistas}
\vspace{1em}

%------------------------------------------------

\NewEntry{Mar 2018}{Exploring the impact of word embeddings for disjoint semisupervised Spanish verb 
sense disambiguation}

\Description{\MarginText{Inteligencia Artificial}{Cristian Cardellino and Laura Alonso Alemany.
\emph{Inteligencia Artificial}, [S.l.], v. 21, n. 61, p. 67-81, mar. 2018. ISSN 1988-3064.}}

%\newpage
\vspace{1em}

%------------------------------------------------

\noindent\spacedlowsmallcaps{Trabajos presentados en congresos}
\vspace{1em}

%------------------------------------------------

\NewEntry{2019}{Convolutional Ladder Networks for Legal NERC and the Impact of 
Unsupervised Data in Better Generalizations}

\Description{\MarginText{FLAIRS}{Cristian Cardellino, Laura Alonso Alemany, Milagro Teruel, 
Serena Villata, and Santiago Marro. In the {\em FLAIRS 2019 - Proceedings of the 32nd
International Florida Artificial Intelligence Research Society Conference}, pp.
155-160, 2019, Sarasota, USA.}}

%------------------------------------------------

\NewEntry{2018}{Increasing Argument Annotation Reproducibility by Using Inter-annotator
Agreement to Improve Guidelines}

\Description{\MarginText{LREC}{Milagro Teruel, Cristian Cardellino, Fernando 
Cardellino, Laura Alonso Alemany and Serena Villata. The {\em 11th International 
Conference on Language Resources and Evaluation} (LREC), 2018, Miyazaki, Japan.}}

%------------------------------------------------

\NewEntry{2017}{Ontology population and alignment for the legal domain:
YAGO, Wikipedia and LKIF}

\Description{\MarginText{ISWC}{Cristian Cardellino, Milagro Teruel, Laura
Alonso Alemany and Serena Villata. At 2017 {\em ISWC Posters and Demonstrations and 
Industry Tracks}, ISWC-P and D-Industry 2017, Vienna, Austria.}}

%------------------------------------------------

\NewEntry{2017}{A low-cost, high-coverage legal named entity recognizer,
classifier and linker}

\Description{\MarginText{ICAIL}{Milagro Teruel, Cristian Cardellino, Laura
Alonso Alemany and Serena Villata. In the proceedings of the {\em 16th 
International Conference on Artificial Intelligence and Law}, ICAIL 2017, 
12 June 2017, pp. 9-18, London, UK.}}

%------------------------------------------------

\NewEntry{2017}{Legal NERC with ontologies, Wikipedia and curriculum learning}

\Description{\MarginText{EACL}{Cristian Cardellino, Milagro Teruel, Laura
Alonso Alemany, and Serena Villata. In the {\em 15th Conference of the European
Chapter of the Association for Computational Linguistics}, EACL 2017,
Proceedings of Conference, Volume 2, pp. 254-259, 2017, Valencia, Spain.}}

%------------------------------------------------

\NewEntry{2017}{Learning slowly to learn better: Curriculum learning for legal
ontology population}

\Description{\MarginText{FLAIRS}{Cristian Cardellino, Milagro Teruel, Laura
Alonso Alemany and Serena Villata. In the {\em FLAIRS 2017 - Proceedings of the 30th
International Florida Artificial Intelligence Research Society Conference}, pp.
252-257, 2017, Marco Island Beach, USA.}}

%------------------------------------------------

\NewEntry{2015}{Improvements in information extraction in legal text by active
learning}

\Description{\MarginText{JURIX}{Cristian Cardellino, Laura Alonso Alemany,
Serena Villata, and Elena Cabrio. In {\em Frontiers in Artificial Intelligence and
Applications}, 279, pp. 21-30, 2015. At the {\em 28th Annual International Conference 
on Legal Knowledge and Information Systems}, JURIX 2015, Braga, Portugal.}}

%------------------------------------------------

\NewEntry{2015}{Information extraction with active learning: a case study in
legal text}

\Description{\MarginText{LNCS}{Cristian Cardellino, Serena Villata, Laura
Alonso Alemany, and Elena Cabrio. {\em Lecture Notes in Computer Science
(including subseries Lecture Notes in Artificial Intelligence and Lecture Notes
in Bioinformatics)}, 9042, pp. 483-494, 2015. At the {\em 16th Annual Conference 
on Intelligent Text Processing and Computational Linguistics}, CICLing 2015, 
Cairo, Egypt.}}

%------------------------------------------------

\NewEntry{2014}{Licentia: a tool for supporting users in data licensing on the
web of data}

\Description{\MarginText{ISWC}{Cristian Cardellino, Serena Villata, Fabien
Gandon, Guido Governatori, Ho-Pun Lam, Antonino Rotolo. {\em CEUR Workshop
Proceedings}, 1272, pp. 277-280, 2014. At the {\em ISWC 2014 Posters and 
Demonstrations Track}, ISWC-P and D 2014, {\em 13th International Semantic 
Web Conference}, ISWC 2014, Riva del Garda, Italy.}}

\vspace{1em}

%------------------------------------------------
%------------------------------------------------

\vspace{1em}

\noindent\spacedlowsmallcaps{Trabajos presentados en simposios}
\vspace{1em}

%------------------------------------------------

\NewEntry{Jul 2017}{In-domain or out-domain word embeddings? A study for legal
cases}

\Description{\MarginText{ESSLLI}{Milagro Teruel and Cristian Cardellino. The
29th {\em European Summer School in Logic, Language, and
Information} Student Session, ESSLLI 2017.}}

%------------------------------------------------

\NewEntry{Sep 2017}{Disjoint semi-supervised Spanish verb sense disambiguation
using word embeddings}

\Description{\MarginText{ASAI}{Cristian Cardellino and Laura Alonso Alemany.
{\em ASAI. $46^{\circ}$ JAIIO}.}}

%------------------------------------------------

\NewEntry{Ago 2015}{A study of semi-supervised Spanish verb sense
disambiguation}

\Description{\MarginText{ESSLLI}{Cristian Cardellino. The 27th {\em European
School of Logic, Language, and Information} Student Session, ESSLLI 2015.}}

%------------------------------------------------

\NewEntry{Jul 2015}{Combining semi-supervised and active learning to recognize
minority senses in a new corpus}

\Description{\MarginText{IJCAI\\Workshop}{Cristian Cardellino, Milagro Teruel,
and Laura Alonso Alemany. {\em Workshop on replicability and reproducibility in
Natural Language Processing: adaptive methods, resources and software}. IJCAI
2015.}}

%------------------------------------------------

\NewEntry{Sep 2013}{SuFLexQA: an approach to Question Answering from the
lexicon}

\Description{\MarginText{ASAI}{Cristian Cardellino and Laura Alonso Alemany.
{\em ASAI. $42^{\circ}$ JAIIO}.}}

%------------------------------------------------
%------------------------------------------------

\vspace{1em}
% \newpage

% \noindent\spacedlowsmallcaps{Trabajos enviados a\'un no aceptados para
% publiaci\'on}\vspace{1em}

% %------------------------------------------------

% \NewEntry{2017}{Exploring the impact of word embeddings for disjoint
% semi-supervised Spanish verb sense disambiguation}

% \Description{\MarginText{Iberamia}{Cristian Cardellino, Laura Alonso i Alemany.
% {\em Edici\'on especial dedicada a ASAI 2017. Revista Iberoamericana de
% Inteligencia Artificial.}}}

% %------------------------------------------------

% \vspace{1em}

\noindent\spacedlowsmallcaps{Asistencia a reuniones cient\'ificas y
tecnol\'ogicas}\vspace{1em}

%------------------------------------------------

\NewEntry{May. 2019}{FLAIRS 2019}

\Description{\MarginText{FLAIRS}{32nd International Florida Artificial Intelligence Research 
Society Conference (FLAIRS). 19 al 22 de Mayo de 2019, Sarasota, EE.UU.}}

%------------------------------------------------

\NewEntry{May. 2018}{LREC 2018}

\Description{\MarginText{LREC}{11th International Conference on Language 
Resources and Evaluation (LREC). 7 al 12 de Mayo de 2018, Miyazaki, Japón.}}

%------------------------------------------------
\newpage

\NewEntry{Nov. 2017}{PyData San Luis (Argentina)}

\Description{\MarginText{PyData San Luis}{PyData San Luis (Argentina) 2017. 
16 al 18 de Noviembre de 2017, San Luis, Argentina.}}

%------------------------------------------------

\NewEntry{Sep. 2017}{CLEI-JAIIO 2017}

\Description{\MarginText{CLEI-JAIIO}{$43^{\circ}$ Conferencia Latinoamericana
de Inform\'atica (CLEI) -- $46^{\circ}$ Jornadas Argentinas de Inform\'atica
(JAIIO). 4 al 8 de Septiembre de 2016, C\'ordoba, Argentina.}}

%------------------------------------------------

\NewEntry{Jul. 2017}{ESSLLI 2017}

\Description{\MarginText{ESLLI}{29th European Summer School on Logic, Language
  and Information. 17 al 28 de Julio de 2017, Toulouse, Francia.}}

%------------------------------------------------

\NewEntry{May. 2017}{FLAIRS 2017}

\Description{\MarginText{FLAIRS}{30th International Florida Artificial Intelligence Research 
Society Conference (FLAIRS). 22 al 24 de Mayo de 2017, Sarasota, EE.UU.}}

%------------------------------------------------

\NewEntry{Ago. 2015}{ESSLLI 2015}

\Description{\MarginText{ESLLI}{27th European Summer School on Logic, Language
  and Information. 3 al 14 de Agosto de 2015, Barcelona, Espa\~na.}}

%------------------------------------------------

\NewEntry{Jul. 2015}{IJCAI 2015}

\Description{\MarginText{IJCAI}{24th International Joint Conference on
  Artificial Intelligence. 25 al 31 de Julio de 2015, Buenos Aires, Argentina.}}

%------------------------------------------------

\NewEntry{Oct. 2014}{ISWC 2014}

\Description{\MarginText{ISWC}{13th International Semantic Web Conference. 19 al
  23 de Octubre de 2014. Riva del Garda, Italia.}}

%------------------------------------------------

\NewEntry{Sep. 2013}{CLEF 2013}

\Description{\MarginText{CLEF}{Conference and Labs of the Evaluation Forum. 23
al 26 de Septiembre de 2013, Valencia, Espa\~na.}}

%------------------------------------------------

\vspace{1em}
%\newpage

\noindent\spacedlowsmallcaps{Seminarios}\vspace{1em}

%------------------------------------------------

\NewEntry{Oct. 2017}{Inteligencia Artificial}

\Description{\MarginText{CUE}{Cristian Cardellino. Jornadas TIC 2017. Colegio
Universitario Ezpeleta. 6 de Octubre de 2017. Morteros. C\'ordoba. Argentina.}}

%------------------------------------------------

\NewEntry{Sep. 2017}{Extracci\'on autom\'atica de argumentos en texto legal}

\Description{\MarginText{FAMAF}{Milagro Teruel y Cristian Cardellino. I
Jornadas de Trabajo en Estudios sobre Procesamiento del Lenguaje. Facultad de
Psicolog\'ia, UNC. 29 de Septiembre de 2017. C\'ordoba. Argentina.}}

%------------------------------------------------

\NewEntry{May. 2017}{Representaci\'on de palabras mediante vectores}

\Description{\MarginText{FAMAF}{Cristian Cardellino. Seminario de Matem\'atica
Aplicada. FAMAFyC, UNC. 11 de Mayo de 2017. C\'ordoba. Argentina.}}

%------------------------------------------------

\NewEntry{Dic. 2014}{De Lenguaje Natural a Linked Data}

\Description{\MarginText{FAMAF}{Cristian Cardellino. Quinta Jornada de
Doctorandos de Computación. FAMAFyC, UNC. 10 de Diciembre de 2014. C\'ordoba.
Argentina.}}

%------------------------------------------------
\newpage
\NewEntry{Nov. 2014}{Licentia: A Tool for Supporting Users in Data Licensing on
the Web of Data}

\Description{\MarginText{INRIA}{Cristian Cardellino. Wimmics Team, INRIA. 28 de
Noviembre de 2014. Sophia Antipolis. Francia.}}

%------------------------------------------------

\NewEntry{Dic. 2013}{Incorporando Inteligencia a la Extracci\'on de
Informaci\'on}

\Description{\MarginText{FAMAF}{Cristian Cardellino. Cuarta Jornada de
Doctorandos de Computación. FAMAFyC, UNC. 2 de Diciembre de 2013. C\'ordoba.
Argentina.}}

%------------------------------------------------

\NewEntry{Oct. 2013}{Subcategorization Frames over Lexicons for Question
  Answering}

\Description{\MarginText{Universit\"at\\Z\"urich}{Cristian Cardellino.
Institut f\"ur Computerlinguistik. 8 de Octubre de 2013. Institut f\"ur
Computerlinguistik. Argentina. Z\"urich. Suiza}}

%------------------------------------------------
%------------------------------------------------

\vspace{1em}
%\newpage

\noindent\spacedlowsmallcaps{Comit\'es}\vspace{1em}

%------------------------------------------------

\NewEntry{2018}{Miembro del comit\'e de organización -- PyData C\'ordoba 2018}

\Description{\MarginText{PyData}{Miembro del comit\'e de organización. Ayuda en la organizaci\'on
de la conferencia y revisi\'on de las publicaciones para PyData C\'ordoba 2018. UNC. 
C\'ordoba, Argentina. Septiembre 2018.}}

%------------------------------------------------

\NewEntry{2018}{Miembro del comit\'e de programa -- MIREL Workshop 2018}

\Description{\MarginText{AICOL}{Revisi\'on de publicaciones para el {\em I
Workshop on Mining and Reasoning with Legal Texts (MIREL)}.
LuxLogAI 2018. Universidad de Luxemburgo, Luxemburgo. Septiembre 2018.}}

%------------------------------------------------

\NewEntry{2016}{Miembro del comit\'e de programa -- AICOL VII}

\Description{\MarginText{AICOL}{Revisi\'on de publicaciones para el {\em VII
Workshop on Artficial Intelligence and Complexity of Legal Systems (AICOL)}.
JURIX 2016. INRIA Sophia Antipolis Mediterran\'ee, Francia. Diciembre
2016.}}

%------------------------------------------------

\NewEntry{2015}{Miembro del comit\'e de programa -- AICOL VI}

\Description{\MarginText{AICOL}{Revisi\'on de publicaciones para el {\em VI
Workshop on Artficial Intelligence and Complexity of Legal Systems (AICOL)}.
JURIX 2015. Universidad de Minho, Braga, Portugal. Diciembre 2015.}}

%------------------------------------------------
%------------------------------------------------

\vspace{1em}
%\newpage

\noindent\spacedlowsmallcaps{Proyectos}\vspace{1em}

\NewEntry{2016-2019}{Mining and reasoning with legal texts}

\Description{\MarginText{MIREL}{El proyecto es financiado por el programa de
investigaci\'on e inovaci\'on {\em European Union's Horizon 2020} bajo la
acuerdo de subvención Marie Sk\l{}odowska-Curie n\'umero 690974}}

% \vspace{2em}
\newpage

%-------------------------------------------------------------------------------
%	FORMACIÓN
%-------------------------------------------------------------------------------

\noindent\spacedlowsmallcaps{\Large{Formaci\'on}}\vspace{1em}

\noindent\spacedlowsmallcaps{Becas Obtenidas}\vspace{1em}

%------------------------------------------------

\NewEntry{2013-2018}{Beca de Posgrado}

\Description{\MarginText{CONICET}{Consejo Nacional de Investigaciones
  Cient\'ificas y T\'ecnicas. Duración: 5 a\~nos.}}

%------------------------------------------------

\NewEntry{2015}{Beca de asistencia ESSLLI 2015}

\Description{\MarginText{EACL}{European Chapter of the Association for
  Computational Linguistics Grant para estudiantes que trabajen en el \'area de
  Lenguaje y Computaci\'on y asistan a ESSLLI 2015.}}

%------------------------------------------------
%------------------------------------------------

\vspace{1em}
%\newpage

\noindent\spacedlowsmallcaps{Cursos de Perfeccionamiento Realizados}\vspace{1em}

%------------------------------------------------

\NewEntry{2016}{Programaci\'on distribu\'ida sobre grandes vol\'umenes de
datos}

\Description{\MarginText{UNC}{Prof. Dr. Damián Barsotti.  Facultad de
Matem\'atica, Astronom\'ia, F\'isica y Computaci\'on.  Universidad Nacional de
C\'ordoba. Argentina.}}

%------------------------------------------------

\NewEntry{2016}{Computaci\'on paralela}

\Description{\MarginText{UNC}{Prof. Dr. Nicol\'as Wolovick. Facultad de
Matem\'atica, Astronom\'ia, F\'isica y Computaci\'on. Universidad Nacional de
C\'ordoba. Argentina.}}

%------------------------------------------------

\NewEntry{2015}{Dise\~no e implementaci\'on de compiladores}

\Description{\MarginText{UNC}{Profs. Marcelo Arroyo y Francisco Bavera.
Facultad de Matem\'atica, Astronom\'ia, F\'isica y Computaci\'on. Universidad
Nacional de C\'ordoba. Argentina.}}

%------------------------------------------------

\NewEntry{2015}{Description logic reasoning}

\Description{\MarginText{RIO}{Prof. PhD. Anni-Yasmin Turhan.
  Escuela de Verano de Ciencias Inform\'aticas RIO 2015.
  Universidad Nacional de R\'io Cuarto. Argentina.}}

%------------------------------------------------

\NewEntry{2014}{Machine learning}

\Description{\MarginText{Coursera}{Prof. PhD. Andrew Ng. Stanford University a
  trav\'es de Coursera.}}

%------------------------------------------------

\NewEntry{2014}{Aceleraci\'on con GPUs: Arquitectura y programaci\'on CUDA}

\Description{\MarginText{RIO}{Prof. Manuel Ujald\'on Mart\'inez.
  Escuela de Verano de Ciencias Inform\'aticas RIO 2014.
  Universidad Nacional de R\'io Cuarto. Argentina.}}

%------------------------------------------------

\NewEntry{2013}{Procesamiento de lenguaje natural}

\Description{\MarginText{UNC}{Prof. Dra. Laura Alonso i Alemany.  Facultad de
Matem\'atica, Astronom\'ia, F\'isica y Computaci\'on. Universidad Nacional de
C\'ordoba. Argentina.}}

%------------------------------------------------

\NewEntry{2013}{Graph-based representation and reasoning in artificial
intelligence}

\Description{\MarginText{ECI}{Prof. PhD. Madalina Croitoru. Escuela de Ciencias
Inform\'aticas 2013. Facultad de Ciencias Extactas y Naturales. Universidad de
Buenos Aires. Argentina.}}

%------------------------------------------------

\NewEntry{2013}{Reductions and causality}

\Description{\MarginText{ECI}{Prof. PhD. Jean-Jacques Levy. Escuela de Ciencias
Inform\'aticas 2013. Facultad de Ciencias Extactas y Naturales. Universidad de
Buenos Aires. Argentina.}}

%------------------------------------------------
%------------------------------------------------

% \vspace{1em}
\newpage

\noindent\spacedlowsmallcaps{Viajes de estudio e investigaci\'on}\vspace{1em}

\vspace{-0.5em}

\Description{\MarginText{13 May. -- 19 Jun. 2019} Visita al grupo WIMMICS-INRIA,
en el marco del proyecto MIREL. Niza. Francia.}

\vspace{-0.5em}

\Description{\MarginText{2 Jul. -- 1 Ago. 2018} Visita al grupo WIMMICS-INRIA,
en el marco del proyecto MIREL. Niza. Francia.}

\vspace{-0.5em}

\Description{\MarginText{1 May. -- 7 Jun. 2018} Visita al grupo WIMMICS-INRIA,
en el marco del proyecto MIREL. Niza. Francia.}

\vspace{-0.5em}

\Description{\MarginText{3 Jul -- 3 Ago. 2017} Visita al grupo WIMMICS-INRIA,
en el marco del proyecto MIREL. Niza. Francia.}

\vspace{-0.5em}

\Description{\MarginText{3 Sep. -- 3 Oct. 2016} Visita al grupo WIMMICS-INRIA,
en el marco del proyecto MIREL. Niza. Francia.}

\vspace{-0.5em}

\Description{\MarginText{2 Jun. -- 2 Dic. 2014} Visita al grupo WIMMICS-INRIA
en pasant\'ia de investigaci\'on y desarrollo. Niza. Francia.}

\vspace{-0.5em}

\Description{\MarginText{10 -- 18 Oct. 2013} Visita a la Universidad de
  Barcelona. Espa\~na.}

\vspace{-0.5em}

\Description{\MarginText{27 Sep. -- 8 Oct. 2013} Visita a la Universidad de
  Z\"urich. Suiza.}

\vspace{2em}
%\newpage

%-------------------------------------------------------------------------------
%	DISTINCIONES
%-------------------------------------------------------------------------------
\spacedlowsmallcaps{\Large{Distinciones o Premios Obtenidos}}\vspace{1em}

\NewEntry{2017}{Most Innovative Application Paper Award}

\Description{\MarginText{ICAIL}{Otorgado por ``International Association for
Artificial Intelligence and Law'', al paper ``Low-cost, High-coverage Legal
Named Entity Recognizer, Classifier and Linker'', en el marco del 16th ICAIL.}}

%------------------------------------------------

\NewEntry{2013}{Premio Universidad}

\Description{\MarginText{UNC}{Diploma con Menci\'on Especial. Universidad
Nacional de C\'ordoba. Argentina.}}

%------------------------------------------------

\vspace{2em} % Extra space between major sections

%-------------------------------------------------------------------------------
%	OTHER INFORMATION
%-------------------------------------------------------------------------------

\spacedlowsmallcaps{\Large{Otros Antecedentes}}\vspace{1em}

%------------------------------------------------

\spacedlowsmallcaps{Vinculaci\'on}\vspace{1em}

%------------------------------------------------

\NewEntry{2018-2019}{\textsc{SANTEX}  --- FAMAFyC. UNC.}

\Description{\MarginText{SANTEX} Docente a cargo del convenio de vinculaci\'on
entre SANTEX y FAMAFyC para el desarrollo en aplicaciones industriales para
Ciencias de Datos.}

%------------------------------------------------
%------------------------------------------------

\vspace{1em}

\spacedlowsmallcaps{Gesti\'on}\vspace{1em}

%------------------------------------------------

\NewEntry{2018-2019}{\textsc{CAC}  --- FAMAFyC. UNC.}

\Description{\MarginText{CAC} Miembro suplente de la Comisi\'on Asesora de Computaci\'on
por el claustro de profesores auxiliares.
Facultad de Matem\'atica, Astronom\'ia, F\'isica, y Computaci\'on. Universidad
Nacional de C\'ordoba. Argentina.}

%------------------------------------------------

\NewEntry{2018-2019}{\textsc{COGRADO}  --- FAMAFyC. UNC.}

\Description{\MarginText{COGRADO} Miembro del Consejo de Grado como
representante de la secci\'on de Computaci\'on.
Facultad de Matem\'atica, Astronom\'ia, F\'isica, y Computaci\'on. Universidad
Nacional de C\'ordoba. Argentina.}

%------------------------------------------------

\NewEntry{2017-2018}{\textsc{CODEPO}  --- FAMAFyC. UNC.}

\Description{\MarginText{CODEPO} Miembro del Consejo de Posgrado como
representante de estudiantes de doctorado por el \'area de Computaci\'on.
Facultad de Matem\'atica, Astronom\'ia, F\'isica, y Computaci\'on. Universidad
Nacional de C\'ordoba. Argentina.}

%------------------------------------------------

\vspace{1em}

\spacedlowsmallcaps{Experiencia Industrial}\vspace{1em}

%------------------------------------------------

\NewEntry{2018}{\textsc{Tappedout.net} --- C\'ordoba. Argentina.}

\Description{\MarginText{Tappedout} Desarrollador web full-stack remoto. 
Ingeniero de aprendizaje autom\'atico.}

%------------------------------------------------

\NewEntry{2014}{\textsc{INRIA}  --- Sophia Antipolis. Francia.}

\Description{\MarginText{INRIA} Pasant\'ia de Investigaci\'on y Desarrollo en
  Scala y Java.}

%------------------------------------------------

\NewEntry{2012}{\textsc{Machinalis}  --- C\'ordoba. Argentina.}

\Description{\MarginText{Machinalis} Desarrollador Jr. Python/Django.}

%------------------------------------------------
%------------------------------------------------

\vspace{1em}
\spacedlowsmallcaps{Idiomas}\vspace{1em}

\newlength{\langbox} % Create a new length for the length of languages to keep them equally spaced
\settowidth{\langbox}{Espa\~nol} % Length equals the length of "English" - if you have a longer language in your list put it here

\Description{\parbox{\langbox}{\textsc{Espa\~nol}}\ \ $\cdotp$\ \ \ \textsc{Ingl\'es}}

% \vspace{-0.5em} % Negative vertical space to counteract the vertical space between every \Description command

% \Description{\parbox{\langbox}{\textsc{Ingl\'es}}\ \ $\cdotp$\ \ \ Flu\'ido}

% \vspace{-0.5em} % Negative vertical space to counteract the vertical space between every \Description command

% \Description{\parbox{\langbox}{\textsc{Franc\'es}}\ \ $\cdotp$\ \ \ B\'asico}

%------------------------------------------------
%------------------------------------------------

% \vspace{1em}

% \spacedlowsmallcaps{Otros}\vspace{1em}

% \Description{\MarginText{Intel} Intel Youth Enterprise Ideation Camp.\\
%   29 de Mayo de 2015. Intel C\'ordoba. Argentina.}

\end{cv}

\end{document}

